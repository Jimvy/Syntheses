\documentclass[fr]{../../../../../../eplexam}
\usepackage{../../../../../../eplcode}

\usepackage{tikz}
\usepackage{xparse}
\usepackage{calc}
%\usepackage{listings}
\usepackage{lipsum}
\usepackage{hyperref}
\lstset{
	language=Java,
	tabsize=4,
%	xleftmargin=3ex,
%	framexleftmargin=3ex,
}

\newlength{\scriptsizetypewriter}
\newlength{\defaultparindent}
%%\NewDocumentEnvironment{newlisting}{%
%%	\setlength{\scriptsizetypewriter}{\widthof{\tiny{x}}}%
%%	\setlength{\defaultparindent}{\parindent}%
%%	\lstset{xleftmargin=\defaultparindent, framexleftmargin=\defaultparindent, numbersep=\scriptsizetypewriter, framesep=0pt}%
%%	\begin{lstlisting}%
%%	}{%
%%	\end{lstlisting}%
%%}%
%\makeatletter
%\let\oldlstlisting\lstlisting
%\let\oldendlstlisting\endlstlisting
%\def\newlisting{\@ifnextchar[\newlisting@i\newlisting@ii}
%\def\newlisting@i[#1]{\oldlstlisting[#1]}
%\def\newlisting@ii{\oldlstlisting}
%\makeatother
\makeatletter
\let\old@lstlisting\lstlisting
\let\old@endlstlisting\endlstlisting
\renewenvironment{lstlisting}{%
\setlength{\scriptsizetypewriter}{\widthof{\tiny{x}}}
\setlength{\defaultparindent}{\parindent}
\lstset{xleftmargin=\defaultparindent, framexleftmargin=\defaultparindent, numbersep=\scriptsizetypewriter, framesep=0pt}
\old@lstlisting
}{%
\old@endlstlisting
}
\makeatother

\hypertitle{info-FSAB1401 % Enter the right title here
}{1}{FSAB}{1401}{2018}{Juin}{All}
{Author1\and Author2\and Author3}
{Professor}

\section{What is the answer to life, the universe and everything?}

\noindent Bonjour

\begin{lstlisting}
public class Main {
	public static void main(String[] args) {
		System.out.println("Hello in my program printing arguments");
		int i = 1;
		for (String arg: args) {
			System.out.println("Hello, here is the " + (i++) + "-th argument: " + arg);
		}
		System.out.println("Goodbye from my program printing arguments.");
		System.exit(0);
	}
}
\end{lstlisting}

\noindent Au revoir

\begin{solution}

\noindent Ajoutons maintenant un petit code suffisamment long que pour causer le bug:
\begin{lstlisting}
public class Main {
	public static void main(String[] args) {
		System.out.println("Hello in my program printing arguments");
		int i = 1;
		for (String arg: args) {
			System.out.println("Hello, here is the " + (i++) + "-th argument: " + arg);
		}
		System.out.println("Goodbye from my program printing arguments.");
		System.exit(0);
	}
}
\end{lstlisting}

Une variante:
\begin{itemize}
	\item Coucou
	\begin{lstlisting}[gobble=4]
	public class Main {
		public static void main(String[] args) {
			System.out.println("Hello in my program printing arguments");
			int i = 1;
			for (String arg: args) {
				System.out.println("Hello, here is the " + (i++) + "-th argument: " + arg);
			}
			System.out.println("Goodbye from my program printing arguments.");
			System.exit(0);
		}
	}
	\end{lstlisting}
\end{itemize}

Et encore plus gros:
\lstinputlisting{MinHeap.java}
\end{solution}

\section{}
Bonjour indenté.

\noindent Bonjour.
%\setlength{\scriptsizetypewriter}{\widthof{\tiny{x}}}
%\setlength{\defaultparindent}{\parindent}
\begin{lstlisting}[firstnumber=987]
print("Hello, world!")
\end{lstlisting}
\begingroup
\setlength\parindent{10em}
\begin{lstlisting}[firstnumber=987]
print("Hello, world!")
\end{lstlisting}
\endgroup
%\begin{lstlisting}[xleftmargin=\defaultparindent, firstnumber=987]
%print("Hello, world!")
%\end{lstlisting}
%\begin{lstlisting}[xleftmargin=2em, framexleftmargin=2em, numbersep=1em, firstnumber=987]
%print("Hello, world!")
%\end{lstlisting}
%\begin{lstlisting}[xleftmargin=2em, framexleftmargin=2em, numbersep=1em, framesep=0pt, firstnumber=987]
%print("Hello, world!")
%\end{lstlisting}
%\begin{lstlisting}[xleftmargin=\defaultparindent, framexleftmargin=\defaultparindent, numbersep=\scriptsizetypewriter, framesep=0pt, firstnumber=987]
%print("Hello, world!")
%\end{lstlisting}
%\begin{lstlisting}[xleftmargin=\parindent, framexleftmargin=\parindent, numbersep=\parindent, framesep=\parindent, firstnumber=987]
%print("Hello, world!")
%\end{lstlisting}
\noindent Au revoir.

Au revoir indenté.

\begin{solution}

\noindent Bonjour.

Bonjour indenté.
%\setlength{\scriptsizetypewriter}{\widthof{\tiny{x}}}
%\setlength{\defaultparindent}{\parindent}
\begin{lstlisting}[firstnumber=987]
print("Hello, world!")
\end{lstlisting}
%\begin{lstlisting}[xleftmargin=\defaultparindent, firstnumber=987]
%print("Hello, world!")
%\end{lstlisting}
%\begin{lstlisting}[xleftmargin=2em, framexleftmargin=2em, numbersep=1em, firstnumber=987]
%print("Hello, world!")
%\end{lstlisting}
%\begin{lstlisting}[xleftmargin=2em, framexleftmargin=2em, numbersep=1em, framesep=0pt, firstnumber=987]
%print("Hello, world!")
%\end{lstlisting}
%\begin{lstlisting}[xleftmargin=\defaultparindent, framexleftmargin=\defaultparindent, numbersep=\scriptsizetypewriter, framesep=0pt, firstnumber=987]
%print("Hello, world!")
%\end{lstlisting}
\noindent Au revoir.

Au revoir indenté.
\end{solution}

\lipsum*[1-3]
Ajoutons une petite figure:
%\begin{solfig}{fig1}{Coucou}
\begin{center}
	\begin{tikzpicture}
	\draw (0, 0) circle (4);
	\end{tikzpicture}
	\captionof{figure}{Coucou}
	\label{fig:fig1}
\end{center}
%\end{solfig}

Et une autre dans un itemize:
\begin{itemize}
	\item Lol
	\begin{center}
	\begin{tikzpicture}
	\draw (0, 0) circle (4);
	\end{tikzpicture}
	\end{center}
\end{itemize}
\lipsum[1-3]

Figure~\ref{fig:fig1}

\end{document}
