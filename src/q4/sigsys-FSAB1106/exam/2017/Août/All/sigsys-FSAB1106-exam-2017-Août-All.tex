\documentclass[en]{../../../../../../eplexam}
\usepackage{../../../../../../eplunits}

\usepackage{mathtools}
\usepackage{tikz}
\usepackage{enumitem}
\usepackage{siunitx}

\newcommand{\rang}{\mathrm{rang}}
\renewcommand{\Im}{\mathrm{Im}}

\allowdisplaybreaks

\usetikzlibrary{shapes,arrows,positioning,calc}
\tikzset{
	block/.style = {draw, fill=white, rectangle, minimum height=3em, minimum width=3em},
	sum/.style = {draw, fill=white, circle, node distance=1cm},
	input/.style = {coordinate},
	output/.style = {coordinate},
	pinstyle/.style = {pin edge={to-, thin, black}}
}

\hypertitle{sigsys-FSAB1106}{4}{FSAB}{1106}{2017}{Août}
{Jean-Martin Vlaeminck}
{Luc Vanderdorpe et Vincent Wertz}

\section{LV1}
Considérons le circuit électrique RLC suivant.
% TODO
On s'intéresse à la relation entre $x(t)$ et $y(t)$. L'équation différentielle qui les lie est
\[ \ffdif{y}{t} + RC \fdif{y}{t} + LC y = \ffdif{x}{t} \]
\begin{itemize}[label=(\alph*)]
	\item Donner la fonction de transfert de ce système.
	\item Donner la réponse impulsionnelle de ce système, pour $R=\SI{15}{\ohm}$, $L=\SI{0.5e-4}{\henry}$ et $C=\SI{1e-3}{\farad}$. % ou 1e-6 ?
	\item Dessiner le diagramme de Bode de ce système pour ces mêmes paramètres.
	\item Si l'on introduit dans le système un signal d'entrée donné par $x(t)=\cos(\omega_1 t + \phi_1) + 0.1 \sin(\omega_2 t + \phi_2)$, qu'obtient-on comme signal de sortie $y(t)$ ?
\end{itemize}

% Solution Grégory Creupelandt :
% L (j\omega) / (R + L j\omega)
% h(t) = delta(t) - 3000 e^{-3000t} u(t)
% ???
% Un graphe avec une droite à +20dB/dec (logiquement) qui passe par -60 en 1 et par 40 en 5, et qui devient alors une droite.
% Le traditionnel ; remarquer la perversité du changement de fréquences, qui complexifie le tout.
% Solution Nicolas Claux, Pauline de Crombrugghe et Basile Cassiers :
% H(j\omega) = \frac{(j\omega)^2}{(j\omega)^2 + \frac{R}{L} j\omega + \frac{1}{LC}}
% h(t) = delta(t) + 1000 (e^{-1000t} -4 e^{-2000t}) u(t)
% Solution Basile Cassiers :
% +40dB/dec jusque 1000 ; +20 jusque 2000 ; 0 après

\section{LV2}
Soient les signaux $x[n]$ et $y[n]$, l'entrée et la sortie d'un système LIT.
\begin{center}
	\begin{tikzpicture}
	\draw[thin, -latex] (-1, 0) -- (4, 0);
	\draw[thin, -latex] (0, -1) -- (0, 2);
	\draw[thick, Green] (0, 0) -- (0, 1) node
	\end{tikzpicture}
\end{center}

% Solution Grégory Creupelandt :
% h[n] = \delta[n] + \delta[n+1]
% H\left(e^{j\Omega}\right) = 1 + e^{-j \Omega}
% y[n] = \cos[n\Omega_0] + \cos[(n-1)\Omega_0]

\section{LV3}
La figure suivante représente le spectre d'un signal échantillonné $x(t)$.
% TODO
% Figure constituée de triangles juxtaposés (façon figure illustrative du livre), avec des zéros aux multiples impairs de 1/T, des 1 aux multiples pairs de 1/T. Il est en fréquentiel habituelle (f)
\begin{itemize}
	\item Le signal de base est passe-bas. Comment peut-on le reconstituer ? Expliquer. Donner le signal reconstruit.
	\item Le signal de base n'est pas passe-bas. Comment peut-on le reconstituer ? Expliquer. Donner le spectre de base sur un dessin, et donner l'expression du signal reconstruit.
\end{itemize}

% Solution Grégory Creupelandt :
% Fuck it

\section{VW1}
Soit la fonction de transfert suivante
\[ H(s) = \frac{s-1}{(s+1)(s-2)} \]
\begin{itemize}[label=(\alph*)]
	\item Quelles sont toutes les régions de convergence possibles pour cette fonction de transfert ?
	\item Pour chacune de ces régions, donner la réponse impulsionnelle du système.
	\item Pour chacune de ces régions, indiquer si elle est stable EBSB et si elle est causale.
\end{itemize}

% Solution Grégory Creupelandt :
% Fonction de transfert (s-1)/((s+1)(s-2))
% Si ROC = s<-1 : h(t) = -2/3 e^{-t} u(-t) - 1/3 e^{2t} u(-t)
%          s> 2 : h(t) = 2/3 e^{-t} u(t) + 1/3 e^{2t} u(t)
%        -1<s<2 : h(t) = 2/3 e^{-t} u(t) - 1/3 e^{2t} u(-t)
% Seul le 3e est BIBO-stable, le 1er et le 3 sont non causaux, le 2 est causal

\section{VW2}
Voir APE stabilité, question avec $H_3 = -H_1/H_2$.

\section{VW3}

Soit le schéma bloc suivant.
%\begin{center}
%	\begin{tikzpicture}[auto, node distance=2cm, >=latex]
%	\node [input, name=input] {$x$};
%	\node [sum, right of=input] (){};
%	\end{tikzpicture}
%\end{center}
% x -----------> + -> y
%    |
%   -1
%    |
%

\begin{itemize}[label=(\alph*)]
	\item Donner une représentation d'état du système.
	\item Étudier la stabilité interne et la stabilité BIBO de ce système discret. Expliquer les différences et critères.
	\item Le système est-il complètement commandable ?
	\item Le système est-il complètement observable ?
\end{itemize}

% Solution Grégory Creupelandt
% (Q_1'; Q_2') = (3/2, -1/2 ; 1, 0) (Q_1 ; Q_2) + (1 ; 0) x
% y = (-3/2, 3/2) (Q_1 ; Q_2) + (-1) x
% Commandable mais non observable

\end{document}
