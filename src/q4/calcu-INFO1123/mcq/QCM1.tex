\begin{mcqs}
  \mcq{Certaines tâches ne peuvent pas être accomplies par un ordinateur}{1}
  {On ne peut pas implémenter les fonction non-calculables.}
  \mcq{Le langage Java est plus puissant que le langage C++}{0}
  {Hypothèse de l'équivalence des langages non-triviaux.}
  \mcq{Il est possible de détecter automatiquement certains virus sur un ordinateur}{1}
  {C'est ce que font les anti-virus,
  ils connaissent certains virus et ceux là ils les détectent.}
  \mcq{Il est possible de détecter automatiquement tout virus sur un ordinateur}{0}
  {Sinon drôle peut être implémenté et ça crée une contradiction.}
  \mcq{Un problème intrinsèquement complexe ne peut pas être résolu, même pour des petites données}
  {0}
  {Non, intrinsèquement complexe ne veut pas dire ``non-calculable'',
  il est exponentiel donc il est très lent mais résoud le problème pour qui sait attendre.}
  \mcq{Grâce à l'évolution technologique, certains problèmes intrinsèquement complexes pourront être résolus dans quelques années}{0}
  {Voir le tableau des améliorations du nombre d'opérations traitées.}
  \mcq{Les problèmes intrinsèquement complexes sont calculables}{1}
  {Par définition, ce sont des problèmes calculable dont tous les
  programmes qui les résolvent ont une complexité au moins exponentielle.}
  \mcq{Les problèmes non calculables sont intrinsèquement complexes}{0}
  {Par définition, un problème intrinsèquement complexe est calculable.}
\end{mcqs}
