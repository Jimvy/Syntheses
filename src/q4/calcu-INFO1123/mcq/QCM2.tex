\begin{mcqs}
  \mcq{L'ensemble des rationnels est énumérable}{1}
  {On peut les mettre dans un tableau 2D et les parcourir en zigzag.}
  \mcq{Un sous-ensemble infini d'un ensemble énumérable est énumérable}{1}
  {C'est un sous-ensemble donc il ne peut pas avoir plus d'éléments.}
  \mcq{Tout ensemble infini de chaînes finies de caractères est énumérable}{1}
  {S'il y a $k$ symbole, la bijection avec les entiers est immédiate si on considère la représentation de ces entiers en base $k$.}
  \mcq{Tout ensemble infini de chaînes infinies de caractères est énumérable}{0}
  {S'il y a $k$ symbole, la bijection avec les réels entre $[0,1]$ est immédiate si on considère la représentation des décimales de ces réels en base $k$.}
  \mcq{L'ensemble des fonctions de $\mathbb{N}$ vers $\{0,1\}$ est non énumérable}{1}
  {Considérons la chaine infinie créée par $f(0)f(1)f(2)\cdots$.
  La bijection avec les réels entre $[0,1]$ est immédiate si on considère leur représentation des décimales de ces réels en binaire.}
  \mcq{L'ensemble des fonctions de $\{0,1\}$ vers $\mathbb{N}$ est énumérable}{1}
  {Représentons ces fonctions par le couple $(f(0),f(1))$.
  $f(0)$ et $f(1)$ sont entiers donc représentés de manière finie.
  Déjà on voit donc qu'on peut représenter les éléments de l'ensemble de manière finie, intuitivement c'est déjà énumérable.
  On voit tout de suite une surjection vers les rationnels $f(0)/f(1)$
  mais ça ne signifie pas qu'on a moins d'éléments que les rationnels car c'est pas injectif $2/2 = 1/1$.
  On peut remarquer cependant que c'est une union dénombrable d'ensemble dénombrables.
  C'est l'union des ensembles
  $\{\, (0,n) \mid n \in \mathbb{N} \,\},
   \{\, (1,n) \mid n \in \mathbb{N} \,\},
   \{\, (2,n) \mid n \in \mathbb{N} \,\},\ldots$.
  Une autre manière de le prouver, c'est en considérant le tableau où $f$ est à la ligne $f(0)$ et la colonne $f(1)$ et en parcourant le tableau en zigzag.}
  \mcq{Tout langage (alphabet fini) est énumérable}{0}
  {Comme les chaines peuven-être infinies. Par contre en informatique les chaines sont finies donc VRAI dans ce cas.}
  \mcq{Toute fonction bijective est injective}{1}
  {Injectif: Ensemble d'arrivée n'est pas la cible de deux éléments de l'ensemble de départ; Bijectif: tous élément est cible de 1 et 1 seul.}
  \mcq{Une fonction dont la table est infinie ne peut être décrite de manière finie}{0}
  {$f\colon \mathbb{R}\rightarrow\mathbb{R} : f(x)=x^{2}$}
  \mcq{Toute fonction totale est surjective}{0}
  {$f\colon \R \to \R : f(x) = x^2$ n'est pas surjective car $\mathop{\mathrm{image}}(f) = \R_+$, les réels positifs
  mais $\mathop{\mathrm{dom}}(f) = \R$.}
  \mcq{Toute extension d'une fonction surjective est surjective}{1}
  {Tout ce qu'on fait c'est parfois changer l'output lorsqu'elle était $\bot$. On a donc au moins toutes les mêmes output qu'avant.}
  \mcq{Tout ensemble non-énumérable peut être mis en bijection avec l'ensemble des réels}{0}
  {Il existe des ensembles plus grands que l'ensembles des réels (cfr. Prof. Deville qui escalade chaises et bancs en CM).}
  \mcq{L'ensemble des fonctions de $\mathbb{N}$ dans $\mathbb{N}$ est énumérable}{0}
  {On peut faire une diagonalisation.
  Chaque fonction est représentée dans une ligne par $f(0),f(1),f(2),\ldots$.
  On modifie simplement la diagonale en faisant $+1$.}
  \mcq{L'ensemble des programmes Java est énumérable}{1}
  {Le code Java est une chaine finie sur un alphabet fini.}
  \mcq{L'énumérabilité des programmes Java et la non énumérabilité des fonctions de $\mathbb{N}$ vers $\mathbb{N}$ est une preuve de l'existence de fonctions non calculables}{1}
  {On peut définir une surjection des programmes vers les fonctions calculables qu'ils calculent.
  C'est une fonction car un programme calcule une seule fonction et c'est surjectif car toutes les fonctions calculables
  ont au moins un programme qui les calcule.
  On conclut qu'il y a un nombre énumérable de fonction calculable, c'est moins que le nombre de fonctions de $\mathbb{N}$ dans $\mathbb{N}$
  qui est non calculable, il y a donc des fonctions non calculables.}
  \mcq{L'ensemble des fonctions calculables est énumérable}{1}
  {Pour chaque fonction calculable, il existe un programme qui la calcule.
  Un programme ne calcule pas 2 fonctions différentes mais 2 programmes différents peuvent calculer une même fonction.
  Il n'y a donc pas plus de fonctions calculables que de programmes.
  Comme le nombre de programme est énumérable,
  il y a donc un nombre énumérable de fonction calculables.}
  \mcq{L'ensemble des fonctions non-calculables est énumérable}{0}
  {L'union de deux ensembles énumérables est énumérables.
  Seulement, l'union de l'ensemble des fonctions calculables et non-calculables
  est l'ensemble des fonctions de $\mathbb{N}$ dans $\mathbb{N}$ qui est non-énumérable.
  Comme l'ensemble des fonctions calculables est énumérable,
  l'ensemble des fonctions non-calculables ne peut pas être énumérable.}
\end{mcqs}
