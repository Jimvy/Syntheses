\subsection{} % Exercice 1
\label{exo:1.1}
If \(A_i\) are countable sets,
\begin{enumerate}
	\item prove that \(A_1 \times A_2\) is a countable set.
	\item prove that \(\bigcup_{i=0}^{+\infty} A_i\)
	is a countable set.
\end{enumerate}

\begin{solution}
	\begin{itemize}
		\item
		\begin{proof}
		We know \(A_1\) and \(A_2\) are countable sets,
		hence we can lay out their elements
		in an infinite table as follows:
		\[
		\begin{array}{c|cccc}
			& A_{11} & A_{12} & A_{13} & \cdots \\
			\hline
			A_{21} & (A_{11}, A_{21}) & (A_{12}, A_{21}) & (A_{13}, A_{21}) & \cdots\\
			A_{22} & (A_{11}, A_{22}) & (A_{12}, A_{22}) & (A_{13}, A_{22}) & \cdots\\
			A_{23} & (A_{11}, A_{23}) & (A_{12}, A_{23}) & (A_{13}, A_{23}) & \cdots\\
			\vdots & \vdots & \vdots & \vdots & \ddots
		\end{array}
		\]
		One can simply zigzag from the top left to the bottom right
		\((A_{11}, A_{21}) \to (A_{12}, A_{21}) \to (A_{11}, A_{22}) \to (A_{11}, A_{23}) \to \cdots\),
		counting all elements of \(A_1 \times A_2\) in the process.
		This means that \(A_1 \times A_2\) is countable.
		\end{proof}
		\item
		\begin{proof}
		We know all sets are countable,
		and that the number of sets being united is countably infinite.
		Hence we can lay everything out in a table as follows.
		\[
		\begin{array}{c|cccc}
			& A_1 & A_2 & A_3 & \cdots \\
			\hline
			1 & A_{11} & A_{21} & A_{31} & \cdots\\
			2 & A_{12} & A_{22} & A_{32} & \cdots\\
			3 & A_{13} & A_{23} & A_{33} & \cdots\\
			\vdots & \vdots & \vdots & \vdots & \ddots
		\end{array}
		\]
		One can simply zigzag from the top left to the bottom right
		\(A_{11} \to A_{21} \to A_{12} \to A_{13} \to \cdots\),
		counting all elements of \(\bigcup_{i=0}^{+\infty} A_i\)
		in the process.
		This means that \(\bigcup_{i=0}^{+\infty} A_i\) is countable.
		To be completely rigorous,
		one needs to take care of duplicates as well.
		\end{proof}
	\end{itemize}
\end{solution}

\subsection{} % Exercice 2
In the following sets, which ones are countable, and which ones are not?
Prove your answers.
\begin{enumerate}
	\item \(\{x \in \N: x \textnormal{ is prime}\}\)
	\item \(\Q\)
	\item \([0, 1] \cap \mathbb{J}\)
	(i.e. the irrational numbers between \(0\) and \(1\)).
	\item \(X^*\) where \(X\) is a countable set
	(\(X^*\) is the set of all finite subsets of \(X\)).
	\item The set of all functions from \(\N\) to \(\{0, 1\}\).
\end{enumerate}

\begin{solution}
\begin{enumerate}
	\item
	\begin{proof}
		This set is countable because it is a subset of \(\N\),
		a countable set.
	\end{proof}
	\item
	\begin{proof}
		This set is countable
		by bijection with \(\Z^2 \setminus \{a, 0\}\).
	\end{proof}
	\item
	\begin{proof}
		By contradiction:
		assume \(\mathbb{J}\) is countable.
		Since we know \(\R = \Q \cup \mathbb{J}\),
		that would mean \(\R\) is countable,
		as we have proved above that \(\Q\) is countable.
		This is absurd, hence \(\mathbb{J}\) is uncountable.
		We are then left with the intersection of two uncountable sets,
		which intersect at an uncountably infinite number of points.
		The set of irrational numbers between \(0\) and \(1\)
		is thus uncountable.
	\end{proof}
	\item
	\begin{proof}
		One can rewrite \(X^*\) as
		\(\bigcup_{i=0}^{+\infty} X^i\),
		where \(X_i\) is the set of subsets of \(X\) of size \(i\).
		Each \(X^i\) is countable, as it is a subset of \(X\),
		a countable set.
		By the proof of Exercise~\ref{exo:1.1},
		this union is also countable.
	\end{proof}
	\item
	\begin{proof}
		Assume this set is countable.
		One can then lay it out in a table,
		as follows:
		\[
		\begin{array}{c|cccc}
			& f_1 & f_2 & f_3 & \cdots \\
			\hline
			1 & f_1(1) & f_2(1) & f_3(1) & \cdots\\
			2 & f_1(2) & f_2(2) & f_3(2) & \cdots\\
			3 & f_1(3) & f_2(3) & f_3(3) & \cdots\\
			\vdots & \vdots & \vdots & \vdots & \ddots
		\end{array}
		\]
		We take the diagonal elements of this table.
		There must exist a function
		\(f_d \colon i \mapsto 1 - f_i(i)\),
		\(f_d \colon \N \to \{0, 1\}\).
		This function is at position \(d\) in the table,
		but upon inspection, we arrive at a contradiction:
		\(f_d(d)\) should be equal to \(1 - f_d(d)\),
		which is impossible
		(given that the functions return either \(0\) or \(1\)).
		This means our initial assumption is wrong,
		and that the set of all functions from \(\N\) to \(\{0, 1\}\)
		is uncountable.
	\end{proof}
\end{enumerate}
\end{solution}

\subsection{} % Exercise 3
Write an algorithm in pseudocode that lists these sets:
\begin{enumerate}
	\item \(\Z\)
	\item \(\{a, b, c\}^*\)
	(all words formed with the alphabet \(\{a, b, c\}\))
	\item The set of all \java{} programs.
\end{enumerate}

\begin{solution}
\begin{enumerate}
\item
\begin{minted}{julia}
function print_Z()
	i = 0
	println(i)
	while true
		i+=1
		println(i)
		println(-i)
	end
end
\end{minted}
\item
\begin{minted}[escapeinside=||]{julia}
function gen(size, prefix, alphabet)
	if size == 0
		println(prefix)
	else
		for c |$\in$| alphabet
			gen(size - 1, prefix * c, alphabet)
		end
	end
end

function print_words()
	i = 0
	while true
		gen(i, "", Set(['a', 'b', 'c']))
		i += 1
	end
end
\end{minted}
\item One can reuse the functions of the previous answer,
but using the set of ASCII characters instead,
and only printing if \mintinline{bash}{javac} succeeds.
\end{enumerate}
\end{solution}

\subsection{} % Exercice 4
Using a cardinality argument,
show that there are functions that are not computable by a \java{} program.

\begin{solution}
\begin{proof}
	The set of all functions is the set of functions from \(\N\) to \(\N\).
	Assume this set, which we will call \(\mathcal{F}\), is countable,
	then all of its subsets must also be countable.
	However, we know that the set of Exercise~1.2.e is not countable,
	while also being a subset of the set of all functions.
	\(\mathcal{F}\) is therefore uncountable and in bijection with \(\R\).

	The set of \java{} programs is a subset\footnote{We only take
	the programs that do not cause compilation errors.}
	of \(\mathcal{S}^*\),
	where \(\mathcal{S}\) is the (countable) set of ASCII characters.
	By the proof of Exercise~1.2.d,
	this means the set of all \java{} programs is countable,
	and in bijection with \(\N\).

	Since the cardinalities of these two sets are not the same,
	there can never be enough \java{} programs to compute all functions.
\end{proof}
\end{solution}

\subsection{} % Exercise 5
True or false?
\begin{enumerate}
	\item The set of functions from \(\N\) to \(\N\)
	has the same cardinality as \(\N\).
	\item The set of finite subsets of \(\N\) is countable.
	\item \(X\) is countable if and only if \(X^*\) is countable.
	\item The set of finite sequences
	from an infinite countable alphabet is countable.
	\item The set of finite sequences from a finite alphabet is countable.
\end{enumerate}

\begin{solution}
\begin{enumerate}
	\item False, as shown in Exercise~1.4,
	the first set is in bijection with \(\R\),
	which does not have the same cardinality as \(\N\).
	\item True, simply apply the argument of Exercise~1.2.d
	to the countable set of natural numbers \(\N\).
	\item True, again by the argument of Exercise~1.2.d.
	\item True, by the proof of Exercise~1.2.d.
	\item False, by a diagonalisation argument.
	Let \(S_i\) be the infinite sequences,
	assume the set is countable.
	Assuming the alphabet contains at least two characters,
	one can then construct a sequence whose \(n\)th character
	is always different
	(for example, the next one in the alphabet, with wrapping)
	from the \(n\)th character of the \(n\)th set.
	This sequence, which should also be in the table,
	say at position \(d\),
	will have a contradiction for its \(d\)th character,
	which should by construction be different from itself.
	This means our assumption is false,
	hence the set of infinite sequences
	from a finite alphabet is uncountable.
\end{enumerate}
\end{solution}
