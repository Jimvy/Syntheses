\documentclass[en]{../../../../../../eplexam}

\hypertitle{Calculabilité, logique et complexité}{4}{INFO}{1123}{2015}{Juin}{All}
{Benoît Legat}
{Yves Deville}

\paragraph{Discussion link}
\url{http://www.forum-epl.be/viewtopic.php?t=13094}

\section{}
Soit
\[ A = \{\, i \mid \exists x : \phi_i(x) = x! \,\}. \]
\begin{enumerate}
  \item Énoncer précisément le théorème de Rice.
  \item Montrer que la non récursivité de $A$ est une conséquence de ce théorème.
\end{enumerate}

\begin{solution}
  \begin{enumerate}
    \item Voir cours.
    \item On voit que les fonction $f(x) = x! + 1$ et $g(x) = 1$ sont calculables. Soit $x$ le numéro de programme de $f$
      et $y$ le numéro de programme de $g$, on voit que $x \notin A$ et $y \in A$.
      L'ensemble $A$ et son complémentaire sont donc non vide.
      Supposons que $A$ est récursif.
      Ça signifie qu'il existe $n \in A$ et $m \in \bar{A}$ tels que
      \[ \phi_n = \phi_m. \]
      Soit $x$ tel que $\phi_n(x) = x!$, on voit que $\phi_m(x) = x!$ donc $m \in A$, absurde!
      $A$ est donc non récursif.
  \end{enumerate}
\end{solution}

\section{}
\begin{mcqs}
  \mcq{L'ensemble des fonctions calculables est énumérable}{1}
  {Voir QCM2.}
  \mcq{Un sous-ensemble infini d'un ensemble récursif est récursif}{0}
  {Voir QCM3.}
  \mcq{La propriété S-m-n affirme que tout numéro de programme calculable peut être transformé en un numéro équivalent, mais avec moins de paramètres.}{0}
  {Voir QCM5.}
  \mcq{Une Machine de Turing dont le ruban serait fini à gauche serait un modèle complet de la calculabilité}{1}
  {Voir QCM8.}
  \mcq{Un formalisme D de calculabilité possède la propriété U (description universelle) lorsque l'interpréteur de D est calculable.}{0}
  {Voir QCM10.}
  \mcq{Tout fonctions calculable peut être calculé par des fonctions primitives récursives.}{0}
  {Voir QCM9.}
  \mcq{Si $A$ est dans $DTIME(n^{2})$, alors $A$ est dans $DSPACE(n^{2})$.}{1}
  {Voir QCM12.}
\end{mcqs}

\section{}
\begin{enumerate}
  \item Énoncer le théorème du point fixe.
  \item Quelle est sa signification ?
\end{enumerate}
\begin{solution}
  Voir cours.
\end{solution}

\section{}
\begin{enumerate}
  \item Définir la réduction polynomiale.
  \item Définir la classe NP.
  \item Qu'est-ce qu'un problème NP-complet.
\end{enumerate}
\begin{solution}
  Voir cours.
\end{solution}

\end{document}
