\section{Axiomes et logique de Hoare}
Dans les exercices suivants, toutes les variables sont de type entier.
On raisonne sur les entiers illimités,
sans tenir compte des problèmes de dépassement.
\subsection{Triplets de Hoare}
Pour rappel, un triplet de Hoare $[P]S[Q]$
est une proposition logique valide ou invalide.
Un triplet est valide si et seulement si,
si $P$ est vrai avant l'exécution de $S$,
alors l'exécution se termine et $Q$ est vrai après.

Déterminez la validité de chacun des triplets suivants
($\texttt{abs}(x)$ calcule la valeur absolue de $x$).
\begin{multicols}{2}
\begin{enumerate}[label={\textbf{\alph*.}}]
	\item $[i = 1]\ \texttt{j := i;}\ [i = j = 1]$.
	\item $[x > 0]\ \texttt{x,y := y,x;}\ [y \ge 0]$.
	\item $[a > 0 \land b < 0]\ \texttt{c := a * b;}\ [c \ge 0]$.
	\item $[0 \le t < N]\ \texttt{t := t+1;}\ [0 < t \le N]$.
	\item $[m = 2 \land n = 10 \land o = m^n]\ \texttt{n := m;}\ [o = 4]$.
	\item $[p = -1]\ \texttt{p := abs(p);}\ [\false]$.
	\item $[p = -1]\ \texttt{p := abs(p);}\ [\true]$.
	\item $[\true]\ \texttt{p := abs(p);}\ [p > 0]$.
\end{enumerate}
\end{multicols}

Soit la fonction $\texttt{fib}(n)$ qui calcule le $n$-ième nombre de Fibonacci
pour $n \ge 0$, et \emph{ne se termine pas} si $n < 0$.
Déterminez la validité de chacun des triplets suivants.

\begin{multicols}{2}
\begin{enumerate}[label={\textbf{\alph*.}}]
	\setcounter{enumi}{8}
	\item $[\true]\ \texttt{p := fib(p);}\ [\true]$.
	\item $[\false]\ \texttt{p := fib(p);}\ [p < 0]$.
	\item $[ab > 0]\ \texttt{p := fib(a+b);}\ [p \ge 0]$.
	\item $[a \ge b]\ \texttt{p := fib(a-b);}\ [p \ge 0]$.
\end{enumerate}
\end{multicols}

\begin{solution}
\begin{enumerate}[label={\textbf{\alph*.}}]
	\item Valide.
	\item Valide (il s'agit d'une affectation simultanée).
	\item Invalide.
	\item Valide (si rien n'est précisé,
	on suppose que les variables sont entières).
	\item Invalide (la valeur de $o$ n'est pas changée).
	\item Invalide.
	\item Valide.
	\item Invalide (ne fonctionne pas si $p = 0$).
	\item Invalide (si $p < 0$, le programme ne termine pas).
	\item Valide.
	\item Invalide (il suffit de prendre $a, b < 0$).
	\item Valide.
\end{enumerate}
\end{solution}

\subsection{Variables auxiliaires}
Les variables auxiliaires, aussi dites fantômes ou rigides,
sont des variables qui n'apparaissent pas dans le programme
en cours de vérification et dont la valeur ne change pas
durant l'exécution de ce dernier.

\begin{enumerate}[label={\textbf{\alph*.}}]
	\item Complétez le triplet suivant afin d'exprimer
	que la valeur de $n$ a été doublée.
	\[
	[?]\ \texttt{n := n * 2;}\ [?]\,.
	\]
	\item Complètez le triplet suivant afin qu'il soit valide,
	qu'il exprime que la valeur de $a + b$ reste inchangée
	et que $b$ ne devient jamais négatif.
	\[
	[?]\ \texttt{a, b := a+1, b-1;}\ [b \ge 0 \land{} ?]\,.
	\]
	\item Concevez un programme qui \og trie \fg{} deux variables $a$ et $b$
	en les modifiant pour que $a$ contienne la plus petite des deux valeurs,
	et $b$ contienne l'autre.
	Bref, pour qu'il réponde à la spécification suivante
	(où $S$ est une variable auxiliaire).
	\[
	[a + b = S]\ \texttt{?}\ [a + b = S \land a \le b]\,.
	\]
	\item La spécification formelle précédente
	n'est pas suffisamment complète par rapport
	à la description en français.
	Trouvez la faille dans les pré/posts formelles, et corrigez-les.
\end{enumerate}

\begin{solution}
\begin{enumerate}[label={\textbf{\alph*.}}]
	\item $[n = N_0]\ \texttt{n := n * 2;}\ [n = 2N_0]$.
	\item $[a + b = S_0 \land b \ge 1]\ \texttt{a, b := a+1, b-1;}\ [b \ge 0 \land a + b = S_0]$.
	\item $[a + b = S]\ \texttt{a, b = min(a,b), max(a,b);}\ [a + b = S \land a \le b]$.
	\item La faille est que
	le code \texttt{a, b = min(a, b) - 1, max(a, b) + 1}
	satisfait également les conditions, mais ne fait pas ce qui est demandé.
	Pour corriger, on pourrait utiliser le triplet suivant:
	\[
	\Big[a + b = S_0 \land a = A_0 \land b = B_0\Big]\ \texttt{...}\ \Big[a+b = S_0 \land a \le b \land \big((a = A_0 \land b = B_0) \lor (a = B_0 \land b = A_0)\big)\Big]\,.
	\]
\end{enumerate}
\end{solution}

\subsection{Affectations}
On donne pour illustration le code suivant qui permute trois variables.
\[
\texttt{a, b, c := b, c, a;}
\]
\begin{enumerate}[label={\textbf{\alph*.}}]
	\item \label{item:3a} Donnez une spécification formelle de ce programme
	sous forme d'un triplet de Hoare valide.
	\item Pour des raisons historiques,
	ce fragment de programme a en fait été implémenté comme suit.
	Annotez ce code à l'aide d'assertions afin d'obtenir un tableau complet
	(contenant au moins une assertion entre chaque instruction).
	Assurez-vous que chaque triplet ainsi formé soit valide.
	\begin{align*}
		&\texttt{a := a + b + c;}\\
		&\texttt{c := a - b - c;}\\
		&\texttt{b := a - b - c;}\\
		&\texttt{a := a - b - c;}
	\end{align*}
	\item Enfin, prouvez que la spécification donné au point~\ref{item:3a}
	est compatible avec votre tableau.
	Quelles sont les implications logiques à prouver pour y parvenir?
\end{enumerate}

\begin{solution}
\begin{enumerate}[label={\textbf{\alph*.}}]
	\item
	\begin{align*}
	&[a = A_0 \land b = B_0 \land c = C_0] \\
	&\texttt{a, b, c := b, c, a;} \\
	&[a = B_0 \land b = C_0 \land c = A_0] \\
	\end{align*}
	\item
	\begin{align*}
		&[a = A_0 \land b = B_0 \land c = C_0] \\
		&\texttt{a := a + b + c;} \\
		&[a = A_0 + B_0 + C_0 \land b = B_0 \land c = C_0] \\
		&\texttt{c := a - b - c;} \\
		&[a = A_0 + B_0 + C_0 \land b = B_0 \land c = A_0] \\
		&\texttt{b := a - b - c;} \\
		&[a = A_0 + B_0 + C_0 \land b = C_0 \land c = A_0] \\
		&\texttt{a := a - b - c;} \\
		&[a = B_0 \land b = C_0 \land c = A_0]
	\end{align*}
	\item L'assistant avait dit que la question était mal posée (ambigüe),
	et qu'on ne savait donc pas y répondre.
\end{enumerate}
\end{solution}

\subsection{Spécifications}
Donnez une spécification formelle pour chacun des problèmes suivants:
\begin{enumerate}[label={\textbf{\alph*.}}]
	\item calculer la taille d'un tableau $a$ donné;
	\item calculer la somme des éléments d'un tableau $a$;
	\item calculer la racine carrée de $x-5$ où $x$ est donné et
	\item calculer le minimum de $x$, $y$ et $z$ et le placer dans $r$.
\end{enumerate}

\begin{solution}
On considère que $\abs{a}$ dénote la taille du tableau $a$,
que $a[i]$ dénote l'élément à la position $i$ de $a$
et que le résultat du programme est toujours stocké dans $r$.
\begin{enumerate}[label={\textbf{\alph*.}}]
	\item
	\begin{align*}
		&[\abs{a} = N_0] \\
		&\texttt{...} \\
		&[r = \abs{a} = N_0]\,.
	\end{align*}
	\item
	\begin{align*}
		&\left[\sum_{i=0}^{\abs{a} - 1} a[i] = S_0\right] \\
		&\texttt{...} \\
		&\left[r = \sum_{i=0}^{\abs{a} - 1} a[i] = S_0\right]\,.
	\end{align*}
	\item
	\begin{align*}
		&\Big[x = X_0\Big] \\
		&\texttt{...} \\
		&\Big[x = \sqrt{X_0 + 5}\Big]\,.
	\end{align*}
	\item
	\begin{align*}
		&[x = X_0 \land y = Y_0 \land z = Z_0] \\
		&\texttt{...} \\
		&[r \le X_0 \land r \le Y_0 \land r \le Z_0 \land (r = X_0 \lor r = Y_0 \lor r = Z_0)]\,.
	\end{align*}
\end{enumerate}
\end{solution}

\subsection{Théorie du problème}
Décrivez précisement les préconditions et postconditions nécessaires
pour spécifier correctement les problèmes suivants.
Si nécessaire, précisez d'abord le problème:
\begin{enumerate}[label={\textbf{\alph*.}}]
	\item trier un tableau d'entiers;
	\item trouver le maximum d'un tableau et
	\item déterminer si un tableau est une permutation d'un autre.
\end{enumerate}

\begin{solution}
\begin{enumerate}[label={\textbf{\alph*.}}]
	\item Pré: $\emptyset$.

	\noindent Post: tout élément du tableau est plus petit ou égal
	à ceux qui le succèdent et le tableau de sortie
	est une permutation du tableau en entrée.
	\item Pré: le tableau est non vide.

	\noindent Post: le résultat est dans le tableau
	et est plus grand ou égal à toutes les entrées du tableau.
	\item Pré: $\emptyset$.

	\noindent Post (première formulation):
	il existe une bijection entre les tableaux.
	Post (seconde formulation): tout élément du premier tableau
	est présent dans l'autre tableau avec la même multiplicité.
\end{enumerate}
\end{solution}

\subsection{}
On a deux tableaux $a$ et $b$ de même taille $\abs{a} = \abs{b} = N$
(indicés de $0$ à $N - 1$).
Soit le programme suivant qui remplit $b$
de sorte que $b[i] = a[0] + a[1] + \cdots + a[i]$.
Écrivez la spécification formelle du programme
(préconditions et postconditions).
\begin{align*}
&\texttt{b[0] := a[0];} \\
&\texttt{var i := 1;} \\
&\texttt{while i < N \{}\\
&\qquad\texttt{b[i] := a[i] + b[i-1];} \\
&\qquad\texttt{i := i + 1;} \\
&\texttt{\}}
\end{align*}

\begin{solution}
Le triplet de Hoare associé au programme est le suivant.
\begin{align*}
&\Bigg[\abs{a} = \abs{b} = N\Bigg] \\
&\texttt{b[0] := a[0];} \\
&\texttt{var i := 1;} \\
&\texttt{while i < N \{}\\
&\qquad\texttt{b[i] := a[i] + b[i-1];} \\
&\qquad\texttt{i := i + 1;} \\
&\texttt{\}} \\
&\Bigg[b[i] = \sum_{j=0}^i a[j]\,, \quad \forall i \in \{\,0, \ldots, \abs{a} - 1\,\}\Bigg]\,.
\end{align*}
\end{solution}
