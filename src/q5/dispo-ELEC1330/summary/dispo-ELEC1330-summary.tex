\documentclass[en]{../../../eplsummary}

\hypertitle{dispo-ELEC1330}{5}{ELEC}{1330}
{Jean-Martin Vlaeminck}
{Vincent Bayot et Denis Flandre}
[\paragraph{Information importante}
Cette synthèse ne contient pour le moment que la matière de la partie dispositifs du cours LELEC1755 - Compléments d'électricité (partie Denis Flandre). Votre aide est la bienvenue pour compléter cette synthèse avec la matière spécifique aux étudiants de majeure.

Les sections qui sont spécifiques au cours de majeure ou de mineure sont signalées avec la mention du cours auquel se rapporte cette section.]

\section{Conduction dans les solides (LELEC1755)}

La conduction dans un milieu donné est due au déplacement de charges (le \emph{courant}) sous l'effet d'un champ électrique, créé par une différence de potentiel électrique ou \emph{tension}.

Dans le vide, la conduction est possible s'il y a des électrons libres. Dans un gaz ionisé, le courant peut provenir des ions chargés positivement ainsi que des électrons chargés négativement. Dans un liquide, les porteurs de charge sont des ions, chargés positivement ou négativement. Dans un solide, on peut trouver tous ces types de porteurs. On distingue
\begin{itemize}
	\item les isolants, qui ne contiennent idéalement aucune charge mobile ;
	\item les conducteurs, qui contiennent un grand nombre de charges mobiles et présentent une faible résistance (peu de chute de tension lorsqu'un courant le traverse) ;
	\item les semi-conducteurs, dont la conductivité est intermédiaire, et peut être modulée par les tensions appliquées.
\end{itemize}

\subsection{Conduction dans les matériaux métalliques}

Dans un métal, les atomes, de grande taille et proches, sont disposés selon une structure régulière \emph{réseau} et forment un cristal. Les électrons de la couche externe ne sont pas attirés par un noyau en particulier, et appartiennent au cristal : ce sont des \emph{électrons libres}, responsables de la conduction.

\end{document}
