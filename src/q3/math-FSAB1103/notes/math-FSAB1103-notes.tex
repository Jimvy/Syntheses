\documentclass[fr]{../../../eplnotes}

\usepackage[SIunits]{../../../eplunits}

\usepackage{pgfplots}
\usepackage{xspace}
\usepackage{amsmath}
\usepackage{amsthm}
\usepackage[frenchb]{babel}

\hypertitle{math-FSAB1103}{3}{FSAB}{1103}
{Jean-Martin Vlaeminck} % narcissique ? Pas vraiment
{Jean-François Remacle, Grégoire Winckelmans et Roland Keunings}[
\paragraph{Remarque importante}
Ce document a été conçu sur base du syllabus sur les équations aux dérivées partielles, et sur les cours magistraux donnés lors de l'année académique 2016-2017. Il a pour objectif de couvrir l'ensemble de la matière utile pour les cours et pour les examens, et d'être plus facile à lire que le syllabus officiel.
]

\newcommand{\fnpart}[3]{\frac{\partial^{#3} #1}{\partial #2^{#3}}}
\theoremstyle{definition}
\newtheorem{defn}{Définition}[section] % mydef
\newtheorem{thm}{Théorème}[section] % mytheo
\newtheorem{lemme}[thm]{Lemme} % mylem
\newtheorem{propriete}[thm]{Propriété} % myprop
\newtheorem{exemple}{Exemple} % myexem

% commandes :
% \fpart{u}{x} donne du/dx avec des partial
% \ffpart{u}{x} donne d^2 u/dx^2 (partial)
% \fdpart{u}{x}{y} donne d^2 u/dx dy (partial)
% \fdif{u}{x} donne du/dx avec des d droits
% \ffdif{u}{x} donne d^2 u/dx^2 (d droits)
% \dif{t} donne dt
% \divn{V} donne la divergence de V
% \rotn{V} donne le rotationnel de V
% \grad{V} donne le gradient
% \lap{V} donne le laplacien

\part{Équations aux dérivées partielles (EDP)}

\section{Introduction et définitions}

Soit une fonction scalaire $u$, dépendant de $m$ variables $x_1, x_2, \ldots, x_m$. Une relation $\mathcal{F}$ entre $u$, les variables $x_i$ et les dérivées partielles de $u$ par rapport à ces variables,
\[ \mathcal{F} \left( u, x_1, \ldots, \fpart{u}{x_1}, \ldots, \ffpart{u}{x_1}, \ldots, \fdpart{u}{x_1}{x_2}, \ldots, \fnpart{u}{x_1}{n} \right) = 0\]
définit une \emph{équations aux dérivées partielles} d'ordre $n$.

La résolution d'une telle équation dépend de plusieurs facteurs et de plusieurs caractéristiques de l'équation, dont l'ordre de l'équation ($n$), sa (non-)linéarité, son homogénéité et encore d'autre facteurs.

\emph{L'ordre} d'une EDP est l'ordre de la dérivée partielle dont l'ordre est la plus élevée. Par exemple, $\fpart{u}{x} + \fdpart{u}{y}{x} -u^3=0$ est d'ordre $2$.

L'équation suivante, avec $A, B, C, D$ et $F$ des fonctions de $x$, $y$ et $u$
\begin{equation}
\label{eq:intro-exemple1}
 A \ffpart{u}{x} + B \fdpart{u}{x}{y} + C \ffpart{u}{y} + Du = F
\end{equation}
est une équation d'ordre 2, à deux variables ; $m=2$ et $n=2$ (mais ce n'est pas le cas en général). Les fonctions $A, B, C, D, F$ sont les coefficients des dérivées partielles. S'il s'agit de fonctions constantes, on parle d'\emph{EDP à coefficients constants}.

Une EDP est dite \emph{linéaire} quand elle l'est par rapport à $u$ et à toutes ses dérivées partielles. Par exemple, si les fonctions $A, B, C, D, F$ de l'équation précédente \ref{eq:intro-exemple1} sont des fonctions de $x$ et $y$, mais pas de $u$ ou d'une de ses dérivées partielles, alors l'équation est linéaire. Une propriété importante des équations linéaires est le \emph{principe de superposition} : si $u(x, y)$ et $v(x, y)$ sont solutions de l'équation, alors toute combinaison linéaire de $u$ et $v$ est aussi solution.

Une EDP est dite \emph{quasi-linéaire} quand elle est linéaire par rapport aux dérivées partielles d'ordre le plus élevé en chacune des variables, c-est-à-dire que les coefficients devant les dérivées partielles d'ordre les plus élevés ne dépendent pas de $u$. Les équations quasi-linéaire se résolvent avec des techniques fort similaires aux techniques utilisées pour les équations linéaires. Elles sont très fréquentes en physique, bien plus que les équations linéaires, et obéissent aux mêmes schémas de \emph{stabilité} numériques que les équations linéaires. Parfois, elles peuvent être écrites sous une forme que l'on qualifie de \emph{forme conservative}.

Une EDP est dite \emph{homogène} quand elle ne contient que des termes ne faisant intervenir $u$ ou ses dérivées partielles. Par exemple, l'équation \ref{eq:intro-exemple1} est homogène si $F=0$. Une équation homogène admet toujours la solution nulle $u=0$. Une EDP linéaire mais non-homogène a comme solution générale la somme d'une solution particulière de l'équation non-homogène et de la solution générale de l'équation homogène correspondante, tout comme leurs cousins EDO.

Enfin, la solution d'une équation aux dérivées partielles, $u=f(x_1, \ldots, x_m)$, est qualifiée de \emph{surface intégrale} ou simplement une intégrale de l'EDP.

\subsection{Quelques exemples, et une remarque concernant les unités}

Afin de mieux comprendre les différents types d'EDP, le mieux est de s'attarder sur quelques exemples, parfois notables.

\[ \ffpart{u}{x} + \ffpart{u}{y} = 0\]
est une EDP à deux variables, d'ordre $2$, linéaire (l'opérateur de dérivation est linéaire), à coefficients constants, et homogène (le membre de droite est nul, et la solution $u=0$ satisfait l'équation). Il s'agit de l'\emph{équation de Laplace}.

\[ x \fpart{u}{x} + y \fpart{u}{y} + \frac{xy}{l^2} u=2u_0 \]
est une équation d'ordre $1$, linéaire (vérifiez !), à coefficients non constants (ce sont des fonctions) et non homogène (le membre de droite est supposé non nul). La présence de la constante $l$ n'est pas anodin : dans le cadre de ce cours, toutes les équations que l'on voit ont une interprétation physique, et ont donc des dimensions et des unités. Ici, $x$ et $y$ ont des unités d'une longueur, et donc $l$ doit avoir les unités d'une longueur également. De même, $u_0$ doit avoir les mêmes dimensions que $u$ (température, pression, chaleur, \ldots).

\[ \left(\fpart{u}{y}\right)^2 \ffpart{u}{x} + \left(\fpart{u}{x}\right)^2 \ffpart{u}{y} = 0 \]
est une équation d'ordre $2$, homogène, non linéaire (à cause des carrés), mais bien quasi-linéaire. En effet, les dérivées partielles d'ordre 2 sont des fonctions linéaires (elles ne sont pas élevées au carré, prises dans un logarithme ou dans un sinus, \ldots) ; le fait que le coefficient de ces dérivées d'ordre 2 dépendent de $u$, voire même de $\fpart{u}{x}$ au carré, ne change pas le fait qu'elle est linéaire.

\[ \fpart{u}{y} \ffpart{u}{x} + \frac{1}{l} \left(\fpart{u}{x}\right)^2 +\fpart{u}{x} \ffpart{u}{y} = 0\]
est également d'ordre $2$, homogène et quasi linéaire (seuls les dérivées d'ordre 2 comptent).

\[ \frac{1}{l} \left(\fpart{u}{x}\right)^2 + \fpart{u}{x} \ffpart{u}{y} = 0 \]
est d'ordre $2$, homogène, mais pas quasi linéaire. En effet, elle est linéaire par rapport à la dérivée partielle la plus élevée en $y$ (qui est $\ffpart{u}{y}$), mais pas par rapport à celle en $x$, qui est $\fpart{u}{x}$.

\[ c \fpart{u}{x} + \fpart{u}{t} = 0 \]
est une EDP à deux variables (une spatiale, $x$, et une temporelle, $t$), d'ordre $1$, linéaire, à coefficients constants si $c$ est une constante, et homogène. L'EDP reste linéaire si $c$ est une fonction de $x$ et/ou de $t$, mais n'est plus linéaire (mais quasi-linéaire) si $c$ est une fonction de $u$. C'est l'\emph{équation de transport}.

\[ u \fpart{u}{x} + \fpart{u}{t} = 0 \]
est une EDP d'ordre $1$, quasi linéaire et homogène. C'est l'\emph{équation de Burgers}, qui peut aussi s'écrire sous une forme conservative,
\[ \fpart{}{x} \left( \frac{u^2}{2} \right) + \fpart{u}{t} = 0 \]

\[ \fpart{}{x} \left( \alpha(u) \fpart{u}{x} \right) - \fpart{u}{t} = 0 \]
est, enfin, une EDP d'ordre 2, quasi-linéaire (ceci peut être vérifié en développant l'équation par la règle du produit) et homogène. Si $\alpha(u) > 0$, il s'agit de l'équation de diffusion, avec $\alpha$ le coefficient de diffusivité.

\section{EDP d'ordre 1}

Commençons donc par résoudre les EDP du 1er ordre. Ces EDP sont dites \emph{à caractère hyperbolique} (on verra plus loin ce que cela signifie). Les équations que nous allons voir dépendent de deux paramètres, $x$ et $y$ (équation dans un plan 2D, indépendante du temps) ou $x$ et $t$ (équation en 1D dépendante du temps), et on a donc, en reprenant les notations de l'introduction, $n=1$ et $m=2$. Dans la suite, on utilisera $x$ et $y$.

Les EDP du 1er ordre quasi linéaires ont la forme
\begin{equation}
\label{eq:order1-genequa}
P \fpart{u}{x} + Q \fpart{u}{y} = R
\end{equation}
avec $P, Q, R$ des fonctions qui dépendent, au plus, de $x$, $y$ et $u$ (elles ne peuvent pas dépendre des dérivées partielles, sinon l'équation ne serait plus linéaire). $R$ peut toujours être écrit sous la forme $R(x, y, u) = F(x, y) + H(x, y, u)$. L'équation sera homogène si $F=0$\footnote{Et très probablement, si $G$ est une fonction linéaire en $u$.}.

Les EDP du 1er ordre linéaires ont presque la même forme, à savoir
\[ P \fpart{u}{x} + Q \fpart{u}{y} + Gu = F \]
où $P, Q, F$ et $G$ sont des fonctions de $x$ et/ou de $y$, mais pas de $u$. Et l'équation est homogène si $F=0$ (dans ce cas, $u=0$ est bien solution). La principale différence entre les EDP quasi-linéaires et linéaires vient du fait que la fonction $R$ s'écrit $R(x, y, u) = F(x, y) - G(x, y)\cdot u$, avec $H(x, y, u) = G(x, y) \cdot u$ ; la fonction $H$ est bien linéaire en $u$.

La solution de cette EDP, notée $u(x, y)$, est alors une surface dans l'espace de dimension 3. Cette surface peut être décrite à partir d'une relation explicite ($u(x, y)$) ou à partir d'une relation implicite ($\mathcal{F}(x, y, u)=0$). Dans la suite, nous considérerons surtout les équations non nécessairement homogènes.

\subsection{Méthode des caractéristiques}

Une EDP seule n'est pas résoluble De la même manière qu'il faut, dans le cas des équations différentielles ordinaires, donner un certain nombre de conditions afin d'obtenir une solution unique, il va falloir donner une sorte de condition initiale afin de pouvoir résoudre le problème. Cette condition initiale sera, dans le cas des EDP à deux variables, une courbe paramétrée $\Gamma(s) \equiv (x(s), y(s))$ sur laquelle on donne la valeur de $u$ : $u(x(s), y(s)) = f(s)$. Le problème de déterminer $u$ sur l'ensemble du domaine, à partir de la connaissance de l'EDP et de la connaissance de la valeur de $u$ sur la courbe $\Gamma$ (nommé \emph{arc de Cauchy}), constitue le \emph{problème de Cauchy}.

Pour résoudre les EDP du 1er ordre, une méthode particulièrement utile et puissance est la \emph{méthode des caractéristiques}. Le principe de cette méthode est la suivante : comme nous connaissons la valeur de $u$ sur $\Gamma$, ainsi que la variation de $u$ dans un voisinage de $\Gamma$ (grâce à l'EDP), il doit être possible de déterminer $u$ dans un voisinage de $\Gamma$, et d'appliquer à nouveau ce principe pour déterminer $u$ en n'importe quel point. On détermine donc $u(x, y)$ de proche en proche, en partant de la courbe $\Gamma$ et en utilisant l'équation différentielle pour obtenir les solutions intermédiaires entre $\Gamma$ et $(x, y)$, ce qui nous permet alors de bien calculer $u$.

Tout le problème devient à présent de déterminer s'il est possible, et à quelles conditions, de déterminer $u$ dans un voisinage d'une courbe, étant donné l'EDP. De même, nous aimerions une méthode générique, dépendant le moins possible de la courbe $\Gamma$ donnée, et de la valeur de $u$ sur cette courbe (la fonction $f(s)$, généralement donnée). Voyons donc ce que l'on peut faire.

Tout d'abord, comme $u(x(s), y(s))=f(s)$ est connu sur la courbe de Cauchy, la dérivée partielle de $u$ par rapport au paramètre $s$ est également connu (c'est $f'(s)$), et donc\footnote{Pour ceux qui se le demanderaient, l'utilisation de la notation avec $\partial$ dans les dérivées partielles est généralement utilisée quand la fonction dépend de plusieurs variables. Si la fonction est à une seule variable, on utilise généralement $\mathrm{d}$ à la place de $\partial$.}

\begin{equation}
f'(s) = \fdif{u}{s} = \fpart{u}{x} \cdot \fdif{x}{s} + \fpart{u}{y} \cdot \fdif{y}{s}
\end{equation}

C'est une équation supplémentaire qui relie $\fpart{u}{x}$ et $\fpart{u}{y}$ ; on peut donc écrire cette équation, avec notre EDP (équation \ref{eq:order1-genequa}), le système
\begin{equation}
\label{eq:order1-bienpose}
\begin{pmatrix}
P & Q \\
\fdif{x}{s} & \fdif{y}{s} \\
\end{pmatrix}
\begin{pmatrix}
\fpart{u}{x} \\
\fpart{u}{y} \\
\end{pmatrix}
= \begin{pmatrix}
R \\
\fdif{u}{s} = f'(s) \\
\end{pmatrix}
\end{equation}

Dans ce système, les fonctions $P$, $Q$, $R$, les dérivées $\fdif{x}{s}$, $\fdif{y}{s}$ et $f'(s)=\fdif{u}{s}$ sont toutes connues. Si le déterminant est non nul, on peut alors toujours déterminer les dérivées partielles $\fpart{u}{x}$ et $\fpart{u}{y}$, et donc déterminer la valeur de $u$ un peu à côté de notre point de référence sur $\Gamma$ ; en effet, du cours de maths 2, on sait que
\[ u(x+\dif{x}, y+\dif{y}) = u(x, y) + \fpart{u}{x}(x, y) \dif{x} + \fpart{u}{y}(x, y) \dif{y} \]

On peut donc déterminer $u$ un peu plus loin de $\Gamma$, et réitérer le processus afin de déterminer $u$ au point qui nous intéresse. Sauf que ce système, et donc cette relation, n'est valable que sur $\Gamma$, ou sur une courbe paramétrée sur laquelle on connait $u$, ce qui n'est pas pratique. Il faut donc trouver une meilleure manière de déterminer cette \og propagation \fg{} de la solution.

Cette méthode ne fonctionne que si le déterminant ne s'annule pas. Si le déterminant s'annule, il est impossible de déterminer les dérivées partielles, et toute la résolution s'écroule. On dira que \emph{le problème de Cauchy est \strong{bien posé}} si le déterminant du système ne s'annule en aucun des points de la courbe. Il est donc important, lorsqu'on pose un problème de Cauchy, de spécifier $\Gamma$ de telle sorte que le problème résultant soit bien posé.

Voyons maintenant ce qu'il se passe si on tente de propager l'équation selon une direction $(\dif{x}, \dif{y})$. Cette direction doit, bien évidemment, être non parallèle à la courbe (sinon on n'avance pas). Autrement dit,
\[ (\dif{x}, \dif{y}) \neq \alpha (\fdif{x}{s}, \fdif{y}{s}) \quad \alpha \in \mathrm{R} \]
La variation de $u$, le long de cette direction, est égale à
\[ \fpart{u}{x} \dif{x} + \fpart{u}{y} \dif{y} = \dif{u} \]
De nouveau, on peut former un nouveau système avec notre EDP,
\begin{equation}
\label{eq:order1-dirpart}
\begin{pmatrix}
P & Q \\ \dif{x} & \dif{y} \\
\end{pmatrix}
\begin{pmatrix}
\fpart{u}{x} \\ \fpart{u}{y} \\
\end{pmatrix}
= \begin{pmatrix}
R \\ \dif{u} \\
\end{pmatrix}
\end{equation}
Afin de simplifier les calculs, nous allons choisir $\dif{x}$ et $\dif{y}$ tels que le déterminant de cette équation s'annule. Cela correspond à une direction particulière, où le calcul de la propagation de $u$ est plus simple. Plus précisément, il s'agit d'une direction telle que
\begin{equation}
\label{eq:order1-caractode}
P\dif{y} = Q \dif{x}
\end{equation}
soit $\frac{P}{Q} = \frac{\dif{x}}{\dif{y}}$, ou encore $(\dif{x}, \dif{y}) \propto (P, Q)$ ; la direction $(\dif{x}, \dif{y})$ est, en quelque sorte, parallèle à la direction définie par $P$ et $Q$.

Le système \ref{eq:order1-dirpart} ne doit pas être confondu avec le système \ref{eq:order1-bienpose}. Ce dernier exprime une condition sur $\Gamma$, telle que le problème est bien posée ; tandis que le système \ref{eq:order1-dirpart} permet de déterminer une direction locale intéressante, parallèle à $(P, Q)$. Cette direction n'est jamais parallèle à la direction de la courbe de Cauchy ; si tel était le cas, le déterminant de \ref{eq:order1-bienpose} serait nul, et le problème serait mal posé\footnote{
	Si $(\dif{x}, \dif{y}) = k \cdot (\fdif{x}{s}, \fdif{y}{s})$, autrement dit s'ils sont parallèles, alors le déterminant de \ref{eq:order1-bienpose} devient \[ \begin{vmatrix} P & Q \\ \fdif{x}{s} & \fdif{y}{s} \\ \end{vmatrix} = \frac{1}{k} \begin{vmatrix} P & Q \\ \dif{x} & \dif{y} \\ \end{vmatrix} =0 \] et donc, le problème serait déjà mal posé à la base.}.

Si le problème est bien posé, alors cette direction particulière n'est jamais parallèle à la courbe $\Gamma$, et définit, en tout point du plan, ce que l'on appelle la \emph{direction caractéristique}. Et en suivant, à partir d'un point donné de $\Gamma$, les différentes directions caractéristiques, on obtient la \emph{courbe caractéristique} : une courbe qui suit, qui est tangente à la direction caractéristique en tout point. On obtient donc, à partir de l'équation $P\dif{y} = Q\dif{x}$, un \emph{réseau de courbes caractéristiques}.

Chaque courbe caractéristique doit croiser une seule fois la courbe de Cauchy $\Gamma$, pour que le problème soit bien posé ; en effet, la courbe caractéristique issue d'un point $P$ sur $\Gamma$ va nous permettre de calculer $u$ sur l'ensemble des points de cette caractéristique. Il est donc important que la courbe caractéristique ne croise qu'une seule fois la courbe de Cauchy ; sinon, on imposerait plusieurs conditions sur cette courbe, ce qui donne généralement des incompatibilités.

Maintenant que l'on connait une direction caractéristique, comment peut-on propager notre solution $u$ le long de cette caractéristique ? Tout simplement en utilisant le système \ref{eq:order1-dirpart}. Comme le déterminant principal s'annule, la seule manière d'avoir encore une solution est que le déterminant formé en remplaçant une colonne de la matrice principale, par la colonne des termes indépendants, s'annule également (un peu à la Kramer). C'est-à-dire :
\[ \begin{vmatrix} P & R \\ \dif{x} & \dif{u} \\ \end{vmatrix} = 0 \quad \quad \begin{vmatrix} R & Q \\ \dif{u} & \dif{y} \\ \end{vmatrix} = 0 \]
\begin{equation}
\label{eq:order1-compatode}
\Leftrightarrow P\dif{u} = R\dif{x} \quad \quad Q\dif{u} = R\dif{y}
\end{equation}

Les deux relations obtenues sont des \emph{relations de compatibilité}, ou relations caractéristiques ; elles définissent le comportement de $u$ qui est compatible avec la direction, et donc la courbe, caractéristique. Elles permettent de relier $\dif{u}$ à la variation $\dif{x}$ le long de la caractéristique. Or, ces relations sont des équations différentielles ordinaires ; notre EDP est donc devenue une EDO le long de chaque caractéristique, ce qui est beaucoup plus simple à résoudre.

Désormais, pour déterminer $u$ à un point $A$ donné, il ne reste plus qu'à intégrer la relation de compatibilité le long de la caractéristique, depuis un point $B$ sur la courbe de Cauchy (où l'on connait $u$) jusqu'à notre point $A$\footnote{Le point $B$ peut être déterminé, vu que s'il y a une caractéristique qui passe par $A$, alors cette caractéristique croise une seule fois la courbe de Cauchy. Il est néanmoins possible que l'on ne puisse pas calculer $u$ à un certain point, si la courbe de Cauchy ou si les caractéristiques empêchent la propagation de $u$ en ce point : caractéristiques absentes, $\Gamma$ trop courte.}.

Il est à noter qu'une seule des deux relations de compatibilité est nécessaire pour l'intégration ; l'autre relation peut directement être obtenue à partir de l'une et de \ref{eq:order1-caractode}. Une manière de retenir ces relations est
\begin{equation}
\label{eq:order1-diffeqnmemo}
\frac{\dif{x}}{P} = \frac{\dif{y}}{Q} = \frac{\dif{u}}{R}
\end{equation}
Néanmoins, comme $P$, $Q$ ou $R$ peuvent s'annuler, il vaut mieux revenir à la forme normale de ces équations lors de l'intégration. C'est essentiellement un moyen mnémotechnique. De même, il vaut mieux choisir la relation la plus simple à intégrer.

Une relation supplémentaire qui peut être utile est $\sqrt{P^2+Q^2} \dif{u} = R\dif{l}$ ; celle-ci relie la variation de $u$ avec la variation $\dif{l}$ directement le long de la caractéristique (dans l'axe de la caractéristique), de sorte que $\dif{l}^2 = \dif{x}^2 + \dif{y}^2$. La démonstration est simplement une règle de la chaine sur $\fdif{u}{l}$, avec quelques manipulations à partir des relation \ref{eq:order1-diffeqnmemo}.

Le cas où $R=0$ donne $\dif{u}=0$ le long de la caractéristique : $u$ est alors constant le long de la caractéristique. On appelle ça un \emph{invariant de Riemann}.

Les EDP du 1\ier{} ordre (avec des fonctions réelles) ont toutes la propriété d'être \emph{à caractère hyperbolique}. Cette propriété peut être résumée, de manière informelle, par le fait qu'elles ont toutes la particularité de pouvoir être résolues par une sorte de propagation de la solution le long de caractéristiques.

Une dernière remarque avant de passer à des exemples. Une certaine EDP du premier ordre, de la forme de l'équation \ref{eq:order1-genequa}, définit un réseau de caractéristiques, via la relation $P\dif{y}=Q\dif{x}$. L'inverse est aussi vrai : un réseau de caractéristiques définit aussi une unique relation de la forme $P\dif{y}=Q\dif{x}$, et donc une EDP du 1\ier{}. Il y a une bijection entre les EDP du 1\ier{} ordre et les réseaux de caractéristiques, et il faut donc être préparé à travailler dans les deux sens (l'examen de septembre 2016 en est un parfait exemple...).

\subsection{Exemples}

Afin d'illustrer la méthode des caractéristiques, considérons l'EDP suivante
\[y\fpart{u}{x} - x\fpart{u}{y} = R\]
et la courbe $\Gamma$ définie par $x(s)=s$ et $y(s)=0$, pour $s$ strictement positif (on verra pourquoi). Il s'agit du demi-axe des x strictement positifs. On spécifie sur cette courbe une fonction $f(s)$ donnée (ici, peu importe sa définition).

On commence par déterminer le réseau des caractéristiques. Puisque $P=y$ et $Q=-x$, les caractéristiques sont déterminées par l'intégration suivante :
\begin{align*}
P \dif{y} &= Q \dif{x} \\
y \dif{y} &= -x \dif{x} \\
\int_{y(s)}^{y} y' \dif{y'} &= \int_{x(s)}^{x} -x' \dif{x'} \\
\int_{0}^{y} y' \dif{y'} &= \int_{s}^{x} -x' \dif{x'} \\
\frac{y^2}{2} &= \frac{s^2}{2} - \frac{x^2}{2} \\
x^2 + y^2 &= s^2
\end{align*}

Comment interpréter cette intégration ? L'idée est que, pour déterminer les points $x$ et $y$ qui seront atteignables par la caractéristique issue d'un point de la courbe de Cauchy en $(x(s), y(s))$ pour un $s$ donné, on va intégrer la direction caractéristique, et donc parcourir la courbe, depuis ce point de la courbe de Cauchy jusqu'à un point $(x, y)$ que l'on suppose appartenir à la caractéristique. Cela donne donc une relation entre $x$ et $y$ pour $s$ donné, et également une relation qui donne $s$ en fonction de $x$ et $y$.

Ici, pour un $s$ donné, la caractéristique n'est rien d'autre que le cercle de rayon $s$ centré à l'origine ; le réseau est donc un ensemble de cercles concentriques. C'est pour ça qu'on a défini la courbe de Cauchy ainsi : la demi-droite ne coupe chaque cercle qu'une seule fois, et n'est pas parallèle à la tangente au cercle. $\Gamma$ est donc admissible et le problème est bien posé.

Ensuite, il ne reste plus qu'à calculer $u$ en chaque point. Pour cela, on utilise une des relations de compatibilité, $P\dif{u} = R\dif{x}$ ou $Q\dif{u}=R\dif{y}$, et on intègre :
\[\int_{u(x(s), y(s))}^{u(x, y)} P \dif{u'} = \int_{x(s)}^{x} R\dif{x'}\]
ou
\[\int_{u(x(s), y(s))}^{u(x, y)} Q \dif{u'} = \int_{y(s)}^{y} R\dif{y'}\]

Considérons le cas où $R=0$. L'équation est alors homogène, et les relations de compatibilité deviennent $\dif{u}=0$ ($P$ et $Q$ ne sont pas nuls en général). Cela signifie que $u$ a la même valeur, partout sur la caractéristique. De fait, si on effectue l'intégrale,
\begin{align*}
\int_{u(x(s), y(s))}^{u(x, y)} P \dif{u'} &= \int_{x(s)}^{x} R\dif{x'} \\
\int_{u(x(s), y(s))}^{u(x, y)} y \dif{u'} &= \int_{x(s)}^{x} 0 \dif{x'} \\
&= 0 \\
\Rightarrow u(x, y) &= u(x(s), y(s))
\end{align*}
Et comme, pour $x$ et $y$ donnés, on a $x^2+y^2=s^2$, on a $s=\sqrt{x^2+y^2}$ et donc,
\[u(x, y) = f(\sqrt{x^2+y^2})\]

\section{Méthode de séparation des variables}

\end{document}
