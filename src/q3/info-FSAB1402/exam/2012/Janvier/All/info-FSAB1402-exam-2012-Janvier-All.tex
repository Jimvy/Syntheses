\documentclass[fr]{../../../../../../eplexam}
\usepackage{../../../info-FSAB1402-exam}

\usepackage{../../../../../../eplcode}

\DeclareMathOperator{\pgcd}{PGCD}

\hypertitle[']{Informatique}{3}{FSAB}{1402}{2012}{Janvier}
{Philippe Verbist \and Louis Devillez}
{Peter Van Roy}

\lstset{language={Oz},morekeywords={for,do}}

\newcommand{\st}{\mathrm{ST}}
\newcommand{\ce}{\mathrm{CE}}
\newcommand{\mozart}{Mozart}

\section{Question 1 : Programmation déclarative (5 pts)}
Pour cette question, vous allez manipuler un jeu de cartes représenté comme une liste de 52 éléments. Chaque élément est un tuple de la forme \lstinline|carte(C V)| ou C 'est la couleur (de 1 à 4 pour piquer, cœur, carreau et trèfle dans cet ordre) et V est la valeur de (1 à 13 pour as jusqu'au roi). Répondez aux points suivants:
\begin{itemize}
	\item Définissez la fonction \lstinline|MakeDeck| qui initialise un jeu de cartes, c'est à dire qui renvoie une liste de 52 cartes différentes quand on l'appelle sans arugments comme \lstinline|J = {MakeDeck}|.
	\item Définissez la fonction \lstinline|J2={Shuffle J1}| pour battre les cartes du jeu \lstinline|J1|. Le résultat \lstinline|J2| contient les mêmes cartes que \lstinline|J1| mais dans un ordre aléatoire. Une possibilité est d'utiliser l'algorithme suivant. Prenez une carte au hasarde de \lstinline|J1| et mettez-là comme première carte de \lstinline|J2|. Continuez avec les cartes qui restent dans \lstinline|J1| jusqu'à l'épuisement de \lstinline|J2|. Vous pouvez supposer que le système connaît la fonction \lstinline|R ={Rand N}| qui renvoie un nombre aléatoire 0 à n -a chaque fois qu'elle est appelée.
	\item Définissez la procédure \lstinline|{Bridge J A B C D}| avec une entrée \lstinline|J| et quatre sorties \lstinline|A,B,C,D|. Cette procédure distribue les cartes aux quatre joueurs e bridge, où à chaque sortie correspond a un joueur. Pour un jeu de 52 cartes, chaque joueur reçoit donc 13 cartes. La procédure doit calculer les cartes que reçoit chaque joueur comme dans un vrai jeu. C'est-à-dire, si \lstinline|J = [C0 C1 C2 |\ldots\lstinline|]| alors \lstinline|A| reçoit \lstinline|C0, B| reçoit \lstinline|C1| et ainsi de suite.
	\item (bonus) Définissez la fonction \lstinline|N={Win A B C D}| qui renvoie un chiffre de 1 à 4 selon le joueur qui gange le pli. Les quatre entrées correspondent au cartes données par chaque joueur dans le pli. L'ordre des couleurs du plus haut au plus bas est pique, coeur, carreau et trèfle. L'ordre des valeurs du plus haut au plus bas est as, roi, dam, valet, 10,\ldots,2.
\end{itemize}
	Faites attention à ce que chaque fonction soit déclarative et récursive terminale (aucune cellule ne sera tolérée !). Essayez de les définir la plus simplement possible et de les écrire avec une belle indentation pour faciliter la lecture. Attention à la syntaxe ! Toute erreur de syntaxe sera sévèrement pénalisée.
	\newpage
\begin{solution}

		\lstinputlisting{2012-01-1.oz}

\end{solution}
\newpage
\section{Question 2 : Sémantique (5 pts)}
Voici un petit programme:

	\lstinputlisting{Q2.oz}

Répondez aux questions suivantes:
\begin{itemize}
	\item Qu'est-ce qui est affiché quand on exécute ce programme ?
	\item Donnez la traduction de ce programme en langage noyau. Attention à donner une traduction complète !
	\item Donnez les environnements contextuels des quatre procédures dans cette traduction.
	\item Donnez quelques pas d'exécution de la machine abstraite pour bien montrer les choses suivantes:
	\subitem La création, la lecture et l'affectation d'une cellule.
	\subitem La définition et l'appel d'une procédure. Qu'est est l'environnement juste avant chaque appel et l'environnement quand on est à l'intérieur de la procédure juste avant l'exécution de son corps ?
\end{itemize}

\begin{solution}
	\begin{enumerate}
		\item 
		\begin{lstlisting}
b
a
		\end{lstlisting}
		Nous avons ensuite une erreur en essayant de faire \lstinline|nil.1|
		\item
	\lstinputlisting{Q2Sol.oz}
		\item Environnement contextuel:
		\subitem NewStack :$\{ \}$
		\subitem Push :$\{ C\rightarrow \alpha\}$
		\subitem Pop :$\{ C\rightarrow \alpha\}$
		\subitem IsEmpty :$\{ C\rightarrow \alpha\}$
		
		\item Machine abstraite (attention il faut utiliser le code donné, ici ce n'est qu'un exemple)
		\begin{enumerate}
			\item
			
			
			$(\{[(\{NewCell\ A\ C\},\{A \rightarrow a,X \rightarrow x,C \rightarrow c\}),\dots ]\},\{a = 1,c,x\},\{\})$
			
			$(\{(X = @C +1,\{A \rightarrow a,C \rightarrow c,X \rightarrow x\})\dots ]\},\{a = 1,c = \xi,x\},\{c:a\})$
			
			$(\{[(C := X,\{A \rightarrow a,C \rightarrow c,X\rightarrow x\})\dots ]\},\{a = 1,c = \xi,x =2\},\{c:a\})$
			
			$(\{[\dots ]\},\{a = 1,c = \xi,x =2\},\{c:x\})$
			
			\item
			$(\{[(F1 = proc\{\$\text{ }X\text{ }R\}R = A + X end,\{A \rightarrow a,F1 \rightarrow f1, B \rightarrow b \}),\dots ]\},\{a = 1,f1\},\{\})$
			
			$(\{[(\{F1\text{ }10\text{ }B\},\{A \rightarrow a,F1 \rightarrow f1,B \rightarrow b\}),\dots ]\},\{a = 1,f1 = (proc\{\$\text{ }X\text{ }R\}R = A + X end, \{A \rightarrow a\}),b\},\{\})$
			
			$(\{[\dots ]\},\{a = 1,f1 = (proc\{\$\text{ }X\text{ }R\}R= A + X end, \{A \rightarrow a\}),b = 11\},\{\})$
			
			
	\end{enumerate}
\end{enumerate}
\end{solution}

\section{Question 3 : Concepts (5 pts)}
Définissez chacun des concepts suivants avec le plus de précision possible. Pour chaque concept donnez un exemple concret (code ou algorithme) pour bien illustrer le concept.
\begin{itemize}
	\item Non déterminisme
	\item Environnement contextuel d'une procédure
	\item Modularité
		\item Problème NP-Complet
	\item polymorphisme
	\item (bonus) La règle de sémantique pour l'instruction \lstinline|local|
\end{itemize}

\begin{solution}
	
	\begin{itemize}
		\item: Non-déterminisme: propriété d'un programme qui peut renvoyer des résultats différents pour des exécutions avec les mêmes paramètres. Le non déterminisme est présent lorsque le système fais des choix lors de l'exécution même lorsque les résultats sont les mêmes.

Exemple:
		\begin{lstlisting}
declare
A = {NewCell 0}
thread A := 1 end
thread A := 2 end
{Browse @A}
		\end{lstlisting}		
		\item Environnement contextuel d'une procédure: Identificateurs utilisés dans le corps de la procédure qui ne sont pas des arguments formels de la procédure et qui ne sont pas non plus définis dans le corps de la procédure.
		
		\begin{lstlisting}
A = 1
fun{Adder X}
   X + A
end
		\end{lstlisting}
		l'environnement contextuel de \lstinline|Adder| est $\{A \rightarrow a\}$
		\item Modularité: On dit qu’un système (ou programme) est modulaire si des mises à jour dans une partie
		du système n’obligent pas de changer le reste. L’état est bénéfique pour la modularité.
		%FIXME: exemple

		\item NP-complet: Un problème est dans la classe NP si on peut vérifier un candidat solution en temps polynomial. Certains problèmes dans la classe NP ont la propriété, que si on trouve un algorithme efficace pour résoudre le problème, on peut dériver un algorithme efficace pour tous les problèmes NP. On les appelle les problèmes NP-Complet.
		
		\item Polymorphisme: Une méthode est polymorphe si elle peut prendre des arguments différents et faire la même chose dans chacun des cas. Une telle méthode est reprise dans la définition d’une interface
		en java.
		
		exemple: browse
		
		\item  La règle de sémantique pour l'instruction \lstinline|local|
		
		\subitem Créer une nouvelle variable x en mémoire
		\subitem Calculer $E' = E + {\langle x\rangle \rightarrow x}$. $E'$ est le même que $E$ sauf qu'il ajoute une correspondance de $\langle x\rangle$ vers $x$
		\subitem Empiler($\langle x\rangle,E'$)
		
		
	\end{itemize}
\end{solution}




\end{document}
